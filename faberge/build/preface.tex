\documentclass[10pt]{article}
\usepackage[utf8]{inputenc}
\usepackage[papersize={17in, 11in}]{geometry}
\usepackage[absolute]{textpos}
\TPGrid[0.5in, 0.25in]{23}{24}
\usepackage{palatino}
\parindent=0pt
\parskip=12pt
\usepackage{nopageno}
\begin{document}

\begin{textblock}{23}(0, 1)
\center \huge PREFACE
\end{textblock}

\begin{textblock}{23}(0, 2.5)

Inscription goes here.

\textbf{Instrumentation:}

\begin{itemize} \itemsep2pt
\item Bass flute (doubling flute)
\item English horn
\item Bass clarinet (doubling B$\flat$ clarinet)
\item Piano
\item Percussion
\item Violin
\item Viola
\item Cello
\end{itemize}


\textbf{Accidentals.} Accidentals govern only one note. This is true even for
successive noteheads at the same staff position. \textit{Because of this no
natural signs appear in the score} (with the exception of parenthesized
noteheads in trills). The sequence of, for example, G$\sharp$4 followed by
G4 (without accidental) is to be understood as G$\sharp$4 followed by
G$\natural$4.

\textbf{The winds are tranposed.} The bass flute sounds an octave lower than
written. The English horn sounds a perfect fifth lower than written. The
B$\flat$ clarinet sounds a major second lower than written and the bass
clarinet sounds a major ninth lower than written.

\textbf{Flute.} Bass flute multiphonics refer to Carin Levine's book
\textit{Die Technik der Flötenspiel}.

\textbf{Piano.} Use a credit card run very slowly laterally up the weaving of
the lowest string of the instrument in the part of the score that requests
individuated clicks.

\textbf{Percussion.} five percussion instruments are required: (1.) marimba;
(2.) bass drum; (3.) crotales in F$\sharp$ and G; (4.) mounted castanets; and
(5.) shakers.

Rolls on the bass drum are all to be as close to attackless as possible: the
rate of the roll doesn't matter but the background depth provided by the
instrument is important.

\textbf{Strings.} Instructions go here.

\end{textblock}

\begin{textblock}{23}(0, 23)

\textbf{Faberg\'{e} Investigations} was written for ensemble recherche who are
to give the world premiere on 21 May 2016 in Paine Hall on the campus of
Harvard University.

\end{textblock}

\end{document}