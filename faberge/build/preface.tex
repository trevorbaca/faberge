\documentclass[10pt]{article}
\usepackage[utf8]{inputenc}
\usepackage[papersize={17in, 11in}]{geometry}
\usepackage[absolute]{textpos}
\TPGrid[0.5in, 0.25in]{23}{24}
\usepackage{palatino}
\parindent=0pt
\parskip=12pt
\usepackage{nopageno}
\begin{document}

\begin{textblock}{23}(0, 1)
\center \huge PREFACE
\end{textblock}

\begin{textblock}{23}(0, 3)

\textit{By the start of the revolution in October 1917, workmasters at the
House of Fabergé had completed fifty jeweled eggs for the last of the Russian
tsars. Forty-three of the eggs survive (auctioned by the Bolsheviks after the
revolution) with the others now lost, or the treasure of thieves. Dazzling in
ribbons of guilloche, the intricacy of the eggs' enamelwork gives to these
objects their ability to transfix: greens, purples, reds made brilliant in
cloisonné. But how to understand the rules of a world that embeds in eggs the
wealth of its kings? How slowly must have gone the work at the jewelers' desks.
And how quickly the changes outside.}

\textbf{Instrumentation:}

\begin{itemize} \itemsep2pt
\item Flute (doubling bass flute)
\item English horn (also playing ratchet)
\item Clarinet in B$\flat$ (doubling bass clarinet)
\item Piano
\item Percussion
    \begin{itemize}
    \item crotales (F$\sharp$6 \& G$\natural$6)
    \item mounted castanets
    \item bass drum
    \item marimba
    \end{itemize}
\item Violin
\item Viola
\item Cello
\end{itemize}

\textbf{Accidentals.} Accidentals govern only one note. This is true even for
successive noteheads at the same staff position. Because of this no natural
signs appear in the score. The sequence of, for example, G$\sharp$4 followed by
G4 (without accidental) is to be understood as G$\sharp$4 followed by
G$\natural$4.

\textbf{The winds are tranposed.} The bass flute sounds an octave lower than
written. The English horn sounds a perfect fifth lower than written. The
B$\flat$ clarinet sounds a major second lower than written and the bass
clarinet sounds a major ninth lower than written.

\textbf{Winds.} Do not tongue changes of pitch: treat all notes as though they
are slurred. 

\textbf{Strings.} No scordatura. The instruction \textbf{spazzolato} (Italian
for ``brushed'' or ``swept'') indicates that the bow should be turned slightly
to the half col legno tratto position (so that the wood of the bow comes into
contact with the string) and then swept alternately up the string (towards the
bridge) and back down again (towards the fingers) according to the rhythms
specified. The instruction \textbf{XFB} (``extremely fast bow'') appears in the
viola part; the music labeled with this indication should be played as an
extremely fast, extremly light, extremely irregular type of tremolo flautando:
use very generous amounts of bow (to create extremely fast bow strokes) and
change the bow frequently in a constantly irregular rhythm. The aural result of
the technique is a ``fluorescent'' type of flautando that brings out the upper
partials of the string's sound, especially as the bow is moved towards the
bridge.

\textbf{The score is a fragment.} The complete \textit{Faberg\'{e}
Investigations} will last about half an hour. The music included (about six
minutes) represents only the opening of the piece.

\end{textblock}

\begin{textblock}{23}(0, 23)

\textbf{Faberg\'{e} Investigations} was written for ensemble recherche, who
gave the world premiere on the 21\textsuperscript{st} of May 2016 in Paine Hall
on the campus of Harvard University.

\end{textblock}

\end{document}